% !TEX root = laws.tex

\section{Property-based testing of a blockchain client}
\label{sec:props}

In this section we discuss our approach to generalized exhaustive testing of an abstract blockchain~(or blockchain-like) protocol implementation. For extensive testing, we test history, minimal state and memory pool separately, and also do thorough checks for node view holder properties.
	
In total, we have implemented 59 property tests. They are using random object generators described in Section~\ref{sec:generators}. Most of the tests are relatively simple, some examples are provided in Section~\ref{sec:simple-props}. Other tests could check complex functionalities where several components are involved. As a particular example, we provide details on testing fork processing in Section~\ref{sec:forking}.

\subsection{Generators}
\label{sec:generators}

We recall that (unlike unit tests, for example), property-based tests are checking not an output of a functionality under test against a concrete input, but rather a relation between input and output values for an arbitrary input value. Thus, in order to run a property-based test, an instance of an input value is needed. To be able to obtain it, a property-based test is supplied with a random input generator, which provides a random input domain element upon request. For our testing framework, a developer of a concrete protocol client needs to provide implementation for generators of the following types:

\begin{itemize}
	\item{a syntactically valid (respectively, invalid) modifier, which is valid (respectively, invalid) against given history instance}
	\item{a semantically valid (respectively, invalid) modifier, which is valid (respectively, invalid) against given minimal state instance. The modifier could be syntactically invalid}
	\item{a totally, so both semantically and syntactically, valid modifier. Respectively, a sequence of totally valid modifiers}
	\item{a transaction}
	\item{history instance, for which it should be should be possible to generate a syntactically valid modifier}
	\item{minimal state instance, for which it should be possible to generate a semantically valid modifier}
	\item{vault instance}
	\item{node view holder instance, for which it should be possible a totally valid modifier}
\end{itemize}

As an example, for the TwinsCoin implementation we provide concrete implementations for all the generators mentioned above. To generate a syntactically valid modifier, we generate a Proof-of-Work block if a previous pair of {\em<Proof-of-Work block, Proof-of-Stake block>} is complete, otherwise we generate a new Proof-of-Stake block. We recall that in TwinsCoin transactions can be recorded only in Proof-of-Stake blocks. A minimal state in the TwinsCoin implementation, similarly to Bitcoin, is defined as a set of current unspent transaction outputs. In order to generate a semantically valid modifier, we generate a Proof-of-Stake block including transactions based on unspent transaction outputs. A totally valid modifier generator, by given a history as well as minimal state instances, is producing whether an empty Proof-of-Work~\footnote{unlike Bitcoin, Twinscoin does not have a notion of a coinbase transaction rewarding miner, instead, block generator's public key is included into the block directly.} or a semantically valid Proof-of-Stake block, depending on a last block in the history~(in order to generate the modifier which is syntactically). 

Interestingly, we implicitly define some properties via generators. In particular, the existence of a generator for a totally valid modifier for any given correct history and valid minimal state instances assumes that it is always possible to make a progress in constructing a blockchain. To the best of our knowledge, the need of this property to be hold was first stated in the formal model of the Bitcoin protocol by Garay et. al.~\cite{Garay2015}~(see ``Input Validity'' definition in Section 3.1 of the paper~\cite{Garay2015}). 

\subsection{Forking}
\label{sec:forking}



We have implemented forking tests for the node view holder. The tests are checking that shorter sequence of totally valid persistent modifiers is not resulting in a fork, a longer sequence leads to a fork. A separate test is trying to apply a fork longer than the maximal rollback depth and checks whether the node view holder is emitting a needed alert.


\knote{remove this section?}
\subsection{Simple properties}
\label{sec:simple-props}

We provide some examples of simple properties we are checking. If a persistent modifier has been appended to a history, it is always possible to get it at some later point of time, for any implementation of the history interface and any syntactically valid modifier. In opposite, if modifier is syntactically invalid, it should be not possible to get it from the history by request. Similarly, if a persistent modifier is semantically valid against a minimal state, the former could be applied to the latter. Even more, after application we can, roll the application back and successfully apply the persistent modifier again.  

For a {\em node view synchronizer} component, which acts as a proxy layer between the node view holder and networking protocol, we check, in particular, that if a transaction or a persistent modifier is coming in from a simulated ``peer'', the node view holder is getting it within a reasonable timeout. Similarly, if a transaction or a persistent modifier is coming from simulated local side, the synchronizer should send it to a network layer within a timeout.
