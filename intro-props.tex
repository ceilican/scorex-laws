% !TEX root = laws.tex

\newcommand{\avector}[2]{(#1_1,#1_2,\ldots,#1_{#2})}
\newcommand{\aDEFvector}[2][a]{(#1_1,#1_2,\ldots,#1_{#2})}

\subsection{Property-Based Testing}
In this section, we give a formal definition of a property followed by a discussion on property-based testing in contrast to conventional testing methodologies.

Within the scope of a data domain $\mathbb{D}$, a property can be seen as a collective abstract behaviour which has to be followed by every valid member of the data domain. More precisely, a property is a predicate $P: \mathbb{D} \rightarrow \{true, false\}$ and it is desirable that it be \emph{valid}: 
\begin{center}
$\forall X \in \mathbb{D}, P(X) = true$
\end{center}
To illustrate, an example of a property $P$ over the domain of pairs of strings $\mathbb{S} \times \mathbb{S}$ is shown below:
\begin{center}
$P((s_1, s_2)) = \#(s_1::s_2) > \#s_1$
\end{center}
where $::$ denotes string concatenation and $\#s$ denotes the length of string $s$. This property is false for any $(s_1, \varepsilon)$, where $\varepsilon$ is the empty string. Therefore, it is not valid.

In contrast to conventional testing methods, where the behaviour of a program is only tested on some pre-determined cases, property-based testing \cite{ron2001property} emphasizes defining properties and then testing their validity against randomly sampled data points. There are various popular libraries available for property testing including QuickCheck for Haskell \cite{claessen2011quickcheck}, JUnit-QuickCheck for Java \cite{jung2015quickcheck}, theft for C, ScalaTest \cite{venners2009scalatest} and ScalaCheck \cite{nilsson2014scalacheck} for Scala.

%In this work we highlight some of these properties which should hold true for any valid implementation of a cryptocurrency.
