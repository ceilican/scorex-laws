% !TEX root = laws.tex

\subsection{Property-based testing}
In the scope of a data domain $\mathbb{D}$, a property can be seen as a collective abstract behaviour which has to be followed by every valid member of that data domain. In other words, a property can be understood as a predicate $P$ on a variable $X$ ($P:X \rightarrow \{true, false\}$) such that: 
\begin{center}
$\forall X \in \mathbb{D}, P(X) = true$
\end{center}
Something here.
Consider an example of a property $P$ that has to be satisfied by the domain of all strings $\mathbb{S}$.
\begin{center}
$\forall X \in \{\#s_1::s_2 >  \#s_1| s_1, s_2 \in \mathbb{S}\}, P(X) = true$
\end{center}
where $::$ denotes a regular string append operation. But if we look closely, we observe that if $s_1, s_2 \in \{\phi\}$ then the property $P$ does not hold true anymore contrary to the fact that both component strings are valid members of $S$. Hence, $P$ is not a valid property over the domain $S$. \\
In contrast to conventional testing methods where it might suffice to test the behaviour of some boundary points at the discrete neighbourhoods, property-based testing emphasises on defining universal (within the domain) properties and then testing their validity against randomly sampled data points. \\
TODO - Generator-driven property testing.