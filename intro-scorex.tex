% !TEX root = laws.tex

\subsection{Scorex Framework}

The idea of a modular design for a cryptocurrency was first proposed by Goodman in Tezos whitepaper~\cite{goodmantezos}. The whitepaper~\footnote{Section 2 of the whitepaper} proposes to split a cryptocurrency design into the three protocols: network, transaction and consensus. In many cases, however, these layers are tightly coupled and it is hard to describe them separately. For example, in a proof-of-stake cryptocurrency a balance sheet structure, which is heavily influenced by a transaction structure, is used in a consensus protocol. To split a cryptocurrency design in more clear way, Scorex 2.0 has finer granularity. In particular, in order to support hybrid blockchains as well as more complicated linking structures than a chain~(such as SPECTRE\cite{EPRINT:SomLewZoh16}), Scorex 2.0 does not have a notion of the blockchain as a core abstraction. Instead, it provides an abstract interface to a \textit{history} which contains \textit{persistent modifiers}. The history is a part of a \textit{node view}, which is a quadruple of $\langle$\textit{history}, \textit{minimal state}, \textit{vault}, \textit{memory pool}$\rangle$. The minimal state is a data structure and a corresponding interface providing an ability to check a validity of an arbitrary persistent modifier for the current moment of time with the same result for all the nodes in the network having the same history. The minimal state is to be obtained deterministically from an inital pre-historical state and the history. The vault holds node-specific information, for example, a node user's wallet. The memory pool holds unconfirmed transactions being propagated across the networks by nodes before got into blocks. 

The whole node view quadruple is to be changed atomically by applying whether a persistent node view modifier or an unconfirmed transaction. Scorex provides guarantees of atomicity and consistency for the application while a coin developer needs to provide implementations for the abstract parts of the quadruple as well as a family of persistent modifiers.