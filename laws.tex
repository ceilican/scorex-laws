\documentclass[]{llncs}   % list options between brackets

\usepackage{color}
\usepackage{graphicx}
%% The amssymb package provides various useful mathematical symbols
\usepackage{amssymb}
%% The amsthm package provides extended theorem environments
%\usepackage{amsthm}
\usepackage{amsmath}

\usepackage{listings}

\usepackage{hyperref}

\usepackage{systeme}

\def\shownotes{1}
\def\notesinmargins{0}

\ifnum\shownotes=1
\ifnum\notesinmargins=1
\newcommand{\authnote}[2]{\marginpar{\parbox{\marginparwidth}{\tiny %
  \textsf{#1 {\textcolor{blue}{notes: #2}}}}}%
  \textcolor{blue}{\textbf{\dag}}}
\else
\newcommand{\authnote}[2]{
  \textsf{#1 \textcolor{blue}{: #2}}}
\fi
\else
\newcommand{\authnote}[2]{}
\fi

\newcommand{\knote}[1]{{\authnote{\textcolor{green}{Alex notes}}{#1}}}
\newcommand{\mnote}[1]{{\authnote{\textcolor{red}{Mayank notes}}{#1}}}


% type user-defined commands here
\usepackage[T1]{fontenc}

\usepackage{xcolor}

\definecolor{dkgreen}{rgb}{0,0.6,0}
\definecolor{gray}{rgb}{0.5,0.5,0.5}
\definecolor{mauve}{rgb}{0.58,0,0.82}

\newcommand{\ma}{\mathcal{A}}
\newcommand{\mb}{\mathcal{B}}
\newcommand{\he}{\hat{e}}
\newcommand{\sr}{\stackrel}
\newcommand{\ra}{\rightarrow}
\newcommand{\la}{\leftarrow}
\newcommand{\state}{state}

\newcommand{\ignore}[1]{} 
\newcommand{\full}[1]{}
\newcommand{\notfull}[1]{#1}
\newcommand{\rand}{\stackrel{R}{\leftarrow}}
\newcommand{\mypar}[1]{\smallskip\noindent\textbf{#1.}\ 
\ 
\
}

\begin{document}

\title{[Short Paper] Checking Laws of the Blockchain With Property-Based Testing}

\author{Alexander Chepurnoy \and Mayank}
\maketitle

\begin{abstract}
Inspired by enormous success of Bitcoin, which reference implementation was released after a whitepaper~\cite{Nakamoto2008} in 2009, many alternative implementations of a Bitcoin protocol, and also alternative blockchain protocols were released. However, implementations contain errors, and cost of an error in case of a cryptocurrency could be extremely high. 

In order to tackle the problem we propose a suite of abstract properties tests checking laws of potentially every blockchain system. 
When developing a new system, a developer needs to instantiate generators for random objects which the tests are using, getting ready test suite checking the laws over many random cases. We provide examples of some laws.
\end{abstract}

% !TEX root = laws.tex

\section{Introduction}

A blockchain-based cryptocurrency is defined by a set of protocols, such as consensus protocol, , agreement on tokens emission rules, . A node, or a {\em client} is software implementation of the protocols. Even more, usually not all the details of all the protocols are specified, so there is some reference client implementation which defines behavior for other implementations. the notable exception here is Ethereum, which Yellow Paper~\cite{ethyp} is trying to define all the details of a client implementation. In Bitcoin, the reference implementation Bitcoin Core is considered to be standard, so an alternative implementation must reproduce its behavior, even bugs. (\knote{cite})

An error in a client implementation would be utterly costly and hard to fix. (\knote{examples}) On other hand, there is an increasing demand in developing more blockchain protocols and clients. 

Plenty of modular open-source frameworks were proposed for speeding up development of new blockchains: Sawtooth~\cite{sawtooth} and Fabric~\cite{fabric} by Hyperledger, Exonum~\cite{exonum} by Bitfury Group, Scorex~\cite{scorex} by IOHK etc. We have chosen Scorex \knote{why?}.


\subsection{Our Contribution}

In this paper we report on design and implementation of a suite of abstract property tests which are implemented for Scorex framework and checking laws of a blockchain client. A developer of a concrete blockchain system just needs to implement generators of random test inputs~(such as blocks and transactions), and then the testing system will extensively check properties against multiple input objects. We have implemented XX~\knote{concrete number} property tests. We integrated the tests into a prototype implementation of the TwinsCoin cryptocurrency, which has two types of blocks. \knote{enhance} 

% !TEX root = laws.tex
\raggedbottom
\subsection{Related Work}

Verification and testing of software systems \cite{myers2011art} is an integral part of a software development lifecycle. Immediately after the implementation of the software, and before its deployment, it has to be verified and tested extensively enough to ensure that all the functional requirements have been properly met. A lot of methods have been developed over the course of time for verification and testing of software. Formal verification \cite{wang2004formal}, for example, is a popular method of program verification~(though testing still remains more prevalent). It is used to validate the correctness of a software module by modeling its behaviour based on a set of formal methods, which are mathematical models specifying the intended functional behaviour. Though there are a lot of formal methods used for this, the most common methods are the ones based on finite state machines \cite{chow1978testing}, Petri Nets, process algebra and timed automata \cite{clarke1996formal}. Prior to formal verification, a mathematical model employing a formal method is decided upon which is then followed by a specification phase where the behaviour of the system is modeled. Finally the actual program is verified against this behaviour specification which is called the verification phase.% In the late 90's, some optimisations were proposed to the otherwise conventional and inefficient exhaustive search methods \mnote{either improve or remove this line}.\cite{holzmann1995improvement}.

Since a software program is developed at module or class level and is integrated with other modules or classes along the development cycle, testing is done at unit level, integration level and system level \cite{myers2011art}, before the software is deployed. End-to-end testing \cite{tsai2001end} is also performed, usually after system testing~(sometimes it is seen as a kind of system testing as well), to validate correct flow spanning different components of the software in real world use cases. Unit tests target individual modules, methods or classes and have a small coverage compared to integration tests which aim towards checking the behaviour of modules when combined together. The two main approaches to unit testing are black box testing and white box testing. The former one focuses on designing test instances without looking inside the code or design, in other words, the black box testing 
is only focusing on the functionality of the unit under testing, while the white box testing approach is more inclined towards testing code coverage i.e. writing test instances which employ the different paths inside the code.

Though initially white box testing was considered as a method suitable for unit testing alone, recently it has emerged as a popular method for integration testing as well. Integration testing is usually done by one or a combination of the following approaches: 
\begin{enumerate}[\IEEEsetlabelwidth{Z}]
\item \textit{Big-Bang approach}.\\In this approach, all the components are integrated together at once and then tested. This method works well for comparatively smaller systems, but is not well suited for larger systems. One obvious disadvantage being that the testing can only begin after all the individual components have been built.
\item \textit{Top-Down approach}.\\As the name suggests, the modules at upper level are tested first and then we move down until we reach the lowest level modules which will be tested at the end. Since lower level modules might not be developed when the upper ones are being tested, stubs are used in place of such the modules. The stubs are trying to simulate behaviour of the modules not implemented yet.
\item \textit{Bottom-Up approach}.\\This approach is opposite to the top-down. Here the lower level modules are tested first and then we iteratively move upwards in the hierarchy until we reach the highest level module. Now as we are testing lower level modules first, stubs are used to simulate the behaviour of higher level modules which may not be implemented yet, if any sibling interaction is required. 
\item \textit{Sandwich approach}.\\ The Sandwich approach is a combination of the Bottom-Up and Top-Down approaches.
\end{enumerate}

Opposed to the conventional unit testing methods which do not take any input parameters, parameterized unit tests~\cite{tillmann2010parameterized} are generalized tests which have an encapsulated collection of test methods whose invocation and behaviour is controlled by a set of input parameters giving more flexibility and automation to unit testing as a whole.

The final full scale testing that a software product undergoes is called the system testing, which includes tests like security test, compatibility test, exceptions handling, scalability tests, stress tests and performance tests. 

In 2013, Visa performed an annual stress test (comes under system testing) to prepare their system VisaNet for the peak traffic of the upcoming holiday season. The test results showed that the system was able to handle 47,000 transactions per second which was around 56\% improvement from the previous year's capacity of the system. %[https://www.visa.com/blogarchives/us/2013/10/10/stress-test-prepares-visanet-for-the-most-wonderful-time-of-the-year/index.html].
Another recent work~\cite{baqer2016stressing} mentions a spam campaign, called \textit{"stress test"}, on the Bitcoin network [Bitcoin cite] caused the network's performance to degrade and essentially resulted in a Denial-of-Service attack, which is cyber-attack on a system where the attack makes the system's resources unavailable or degrades their intended quality to a point where it becomes difficult or sometimes impossible for the honest users to avail the resource. The intention behind this campaign was to expose the vulnerabilities of the network, particularly to spam attacks, and to therefore increase the present transaction verification rate of the network. The authors of the paper~\cite{baqer2016stressing} present an experimental analysis of the above mentioned "attack" by using $k$-means clustering along with some specific features to differentiate between spam and actual transactions. They report that around 23\% of the total transactions flowing on the network were indeed spams during the peak period of the attack. Since this number is a pretty significant number and potentially degrade the network by at least a quarter. As a consequence of this attack, a special type of transactions were observed, called $UTXO$-cleanup transactions, which were created by miners to combine spam transaction together, reducing the $UTXO$ set size, and hence reduce the impact of the attack on the network. 

% ignored text begins
\ignore{ Similar to this, another vulnerability in the Bitcoin system caused MtGox Bitcoin exchange to close in February 2014 [?https://www.businessinsider.in/Bitcoin-Just-Completely-Crashed-As-Major-Exchange-Says-Withdrawals-Remain-Halted/articleshow/30165462.cms]. The exchange announced that close to 850,000 bitcoins were stolen by an attacker who exploited the vulnerability that causes bitcoin transactions to be malleable. Let us denote a bitcoin transaction as the tuple $T = (M, sig$) where $M$ is the message content of transaction and $sig$ is a valid signature on $M$. If a transaction is non-malleable, then it is not viable to construct another transaction $T' = (M, sig')$ such that $sig'$ is also a valid signature on $M$, without the knowledge of the secret key. Due to the fact that in bitcoin a transaction is identified by its unique $id = \mathbb{H}(M, sig)$, where $\mathbb{H}$ is a hash function, and not just $id = \mathbb{H}(M)$ means that $T$ and $T'$ as mentioned above will be treated as different transactions since they will have different $id$, despite the fact that their transaction content is exactly same. The above malleability is possible due to the fact that signature schemes used in bitcoin can be malleable. The way this attack was used to steal money, as claimed by the exchange, is the following:
\begin{enumerate}[\IEEEsetlabelwidth{Z}]
\item A user begins by depositing a certain amount $a$ into exchange's account.
\item He then asks the exchange to transfer his money back to him.
\item The exchange issues a transaction $T$ to transfer $a$ bitcoins to the user's account.
\item The user constructs another transaction $T'$ by exploiting the malleability of transactions.
\item Suppose that somehow $T'$ gets included in the blockchain instead of $T$.
\item This ensures that the user gets $a$ bitcoins in his account. But after this, the user files a request for resending the money claiming that he didn't receive it.
\item To respond to this request, the exchange checks that no transaction with $id = id_T$ ($=\mathbb{H}(T)$) is present and reissues another transaction sending $a$ to the user. This way the user was able to receive double that coins that he were to receive without the vulnerability.
\end{enumerate}
A very intuitive solution [?https://eprint.iacr.org/2013/837.pdf] to this problem is to change bitcoin such that transactions are identified only with $M$ and not the input scripts (signatures). This would mean that even if a signature is forged, the new transaction will hash to the same $id$ as the previous transaction and would eliminate this issue. There is also another solution [?%https://fc15.ifca.ai/preproceedings/bitcoin/paper_9.pdf
] where malleability of bitcoin transaction is dealt-with specific to bitcoin contracts. }
% ignored text ends

These arguments make it pretty convincing that repeated testing of even the most carefully written and designed system is crucial to expose hidden vulnerabilities in the developed system which might miss the eye of the developers. Such tests are performed regularly on important system and help ensure their reliability, security and performance. But, if we talk about Bitcoin and other blockchain system, where the cost of errors can be humungous, then testing becomes the most important part of the software development lifecycle.

In this manuscript we focus on a different method of program testing, called property testing \cite{ron2001property}. This method is concerned with making approximate decisions on whether a function under test satisfies a property globally or not by using only a small number of random input domain elements. We argue that for testing an implementation on top of an abstract framework, like Scorex, this method is of particular interest, since we can split clearly and in a useful way work between the framework and the application. Concretely, properties an application should satisfy are provided before implementing it, and then an application developer just needs to write generators for random input domain elements the tests are using.   
\nocite{holzmann1995improvement}
\nocite{zaki2008formal}

%Some papers to use for references particular to property testing - http://www.wisdom.weizmann.ac.il/~oded/test.html.%
%Integration testing papers for references - https://www.researchgate.net/profile/Xiaoying_Bai/publication/221028427_End-To-End_Integration_Testing_Design/links/02e7e516cabf5c969d000000.pdf%
%Formal verification - http://www.cerc.utexas.edu/~jay/fv_surveys/zaki-AMS-survey-FULL.pdf%
%Formal verification - http://www.cerc.utexas.edu/~jay/fv_surveys/wang_fvsurvey_timed_systems_proc_ieee2004.pdf%
%Unit testing - https://link.springer.com/article/10.1007/s10664-006-5964-9%
%Integration testing - https://link.springer.com/chapter/10.1007/978-3-540-31862-0_18%
%http://citeseerx.ist.psu.edu/viewdoc/download?doi=10.1.1.93.7961&rep=rep1&type=pdf%
%http://ieeexplore.ieee.org/document/1702519/%
%https://dl.acm.org/citation.cfm?id=1767341%




% !TEX root = laws.tex

\newcommand{\avector}[2]{(#1_1,#1_2,\ldots,#1_{#2})}
\newcommand{\aDEFvector}[2][a]{(#1_1,#1_2,\ldots,#1_{#2})}

\subsection{Property-based testing}
In this section, we will begin by giving a formal definition of a property followed by a discussion on property-based testing in contrast to the conventional testing methodologies.\\
Within the scope of a data domain $\mathbb{D}$, a property can be seen as a collective abstract behaviour which has to be followed by every valid member of that data domain. In other words, a property can be understood as a predicate $P$ over a variable $X$ ($P:X \rightarrow \{true, false\}$) such that: 
\begin{center}
$\forall X \in \mathbb{D}, P(X) = true$
\end{center}
Let us consider an example of a property $P$ over the domain of all possible strings $\mathbb{S}$.
\begin{center}
$\forall X \in \{s_1::s_2 | \#s_1::s_2 >  \#s_1 \wedge s_1, s_2 \in \mathbb{S}\}, P(X) = true$
\end{center}
where $::$ denotes a regular string append operation and $\#s$ denotes the length of string $s$. But if we look closely, we observe that if $s_1, s_2 \in \{\phi\}$ then the property $P$ does not hold true anymore contrary to the fact that both component strings are valid members of $S$. Hence, $P$ is not a valid property over the domain $S$. \\
In contrast to conventional testing methods where it might suffice to test the behaviour of some boundary points at the discrete neighbourhoods, property-based testing emphasises on defining universal (within the domain) properties and then testing their validity against randomly sampled data points. There are some popular libraries available for property testing including QuickCheck for Haskell, JUnit-QuickCheck for Java, theft for C, and ScalaTest and ScalaCheck (majorly for generator-driven property testing) for Scala-based programs. It is trivial to understand that in order to automate the process of program testing at scale, having properties defined for all the different components of the program largely optimises the whole process. \\
Since it might not be the case everytime that the random data points needed for property testing are indeed the primitive data types, so in such cases generators are defined, which are simple functions that usually generate some output based on input parameters, to facilitate data generation for testing. A simple example of a generator $G$ can be:
\begin{center}
$G(X) \rightarrow {Y}$, where $X=\aDEFvector[x]{n+1}$ and $Y=\{x | x \% 2 = 0\}$
\end{center}
Generators can automate the process of random data point generation on virtually any data type, which can then be verified against the respective properties to yield the data points, if any, which falsify the predicate.\\

\knote{improve the bridge}
Walking inside the walls of bitcoin and other cryptocurrencies, we, in this work, highlight some of these properties which should hold true for any implementation of a valid cryptocurrency.
% !TEX root = laws.tex

\subsection{Scorex Framework}

The idea of a modular design for a cryptocurrency was first proposed by Goodman in Tezos whitepaper~\cite{goodmantezos}. The whitepaper~\footnote{Section 2 of the whitepaper} proposes to split a cryptocurrency design into the three protocols: network, transaction and consensus. In many cases, however, these layers are tightly coupled and it is hard to describe them separately. For example, in a proof-of-stake cryptocurrency a balance sheet structure, which is heavily influenced by a transaction structure, is used in a consensus protocol. To split a cryptocurrency design in clearer way, Scorex 2.0 has finer granularity. In particular, in order to support hybrid blockchains as well as more complicated linking structures than a chain~(such as SPECTRE\cite{EPRINT:SomLewZoh16}), Scorex 2.0 does not have the notion of blockchain as a core abstraction. Instead, it provides an abstract interface to a \textit{history} which contains \textit{persistent modifiers}. The history is a part of a \textit{node view}, which is a quadruple of $\langle$\textit{history}, \textit{minimal state}, \textit{vault}, \textit{memory pool}$\rangle$. The minimal state is a data structure and a corresponding interface providing the ability to check the validity of an arbitrary persistent modifier for the current moment of time with the same result for all the nodes in the network having the same history. The minimal state is to be obtained deterministically from an initial pre-historical state and the history. The vault holds node-specific information, for example, a node user's wallet. The memory pool holds unconfirmed transactions being propagated across the networks by nodes before their inclusion into blocks. \knote{write what are we doing with it}
% !TEX root = laws.tex

\section{Blockchain properties}
\label{sec:props}

In this section we discuss our approach to generalized testing of an abstract blockchain-like system. For extensive testing, we test history, minimal state and memory pool separately, and also do thorough checks for node view holder properties.
	
In total, we have implemented 59 property tests. They are using random object generators described in Section~\ref{sec:generators}. Most of the tests are relatively simple, some examples are provided in Section~\ref{sec:simple-props}. Other tests could check complex functionalities where several components are involved. As a particular example, we provide details on testing fork processing in Section~\ref{sec:forking}.

\subsection{Generators}
\label{sec:generators}

In our test suites we use generators for objects which are used to check the laws of a blockchain. A developer of a concrete system needs to implement these generators. We provide interfaces for generators of the following types of objects:

\begin{itemize}
	\item{a syntactically valid (respectively, invalid) modifier, which is valid (respectively, invalid) based on the node's local view of history.}
	\item{a semantically valid (respectively, invalid) modifier, which is valid (respectively, invalid), based on the node's local view of minimal state.}
	\item{a totally, so both semantically and syntactically, valid modifier. Respectively, a sequence of totally valid modifiers}
	\item{a transaction}
\end{itemize}

As an example, for the TwinsCoin implementation we provide concrete implementations for all the generators mentioned above. To generate a syntactically valid modifier, we generate a Proof-of-Work block if a previous pair of {\em<Proof-of-Work block, Proof-of-Stake block>} is complete, otherwise we generate a new Proof-of-Stake block. It can be noted that, in TwinsCoin, transactions are only recorded in Proof-of-Stake blocks. A minimal state in our TwinsCoin implementation, similarly to Bitcoin, is defined as a set of current unspent transaction outputs. To generate a semantically valid modifier, we generate a Proof-of-Stake block including transactions based on unspent transaction outputs. A totally valid modifier generator has a matching Proof-of-Stake block among a syntactically valid and a semantically valid modifier. 

Interestingly, we implicitly define some properties via generators. In particular, the existence of a generator for a totally valid modifier for any given correct history and valid minimal state assumes that it is always possible to make a progress in constructing a blockchain.

\subsection{Simple properties}
\label{sec:simple-props}

We provide some examples of simple properties we are checking. If a persistent modifier has been appended to a history, it is always possible to get it at some later point of time, for any implementation of the history interface and any syntactically valid modifier. In opposite, if modifier is syntactically invalid, it should be not possible to get it from the history by request. Similarly, if a persistent modifier is semantically valid against a minimal state, the former could be applied to the latter. Even more, after application we can, roll the application back and successfully apply the persistent modifier again.  

For a {\em node view synchronizer} component, which acts as a proxy layer between the node view holder and networking protocol, we checks, in particular, that if a transaction or a persistent modifier is coming in from a simulated ``peer'', the node view holder is getting it within a reasonable timeout. Similarly, if a transaction or a persistent modifier is coming from simulated local side, the synchronizer should send it to a network layer within a timeout.

\subsection{Forking}
\label{sec:forking}

Processing forks could be a complicated issue, making testing of this functionality important. We proceed by describing the way in which forking is implemented in Scorex. When a persistent modifier is appended to the history, it returns~(if the modifier is syntactically valid) {\em progress info} structure which contains a sequence of persistent modifiers to apply as well as a possible identifier of a modifier to rollback~(for the minimal state, vault, memory pool) before the application of the sequence. For efficiency reasons, the minimal state is usually limited in maximal depth for a rollback, so the rollback could fail~(this situation is probably unresolvable in a satisfactory way without a human intervention). 

We have implemented forking tests for the node view holder. The tests are checking that shorter sequence of totally valid persistent modifiers is not resulting in a fork, a longer sequence leads to a fork. A separate test is trying to apply a fork longer than the maximal rollback depth and checks whether the node view holder is emitting a needed alert.

% !TEX root = laws.tex

\section{Examples of properties tests}
\label{sec:examples}

To explain our approach to the testing of a client in details, in this section we provide some examples of property tests which are valid for most of blockchain-based systems. We have grouped the tests based on their similarity.

\begin{enumerate}[\IEEEsetlabelwidth{Z}]
\item \textit{Memory pool Tests}.\\
Memory Pool (or just mempool in the Bitcoin jargon) is used to store unconfirmed transactions which are to be included into persistent modifiers. The following tests are used to check some general properties of a memory pool which every blockchain client should pass.

\begin{itemize}[\IEEEsetlabelwidth{Z}]

\item \textit{A memory pool should be able to store enough transactions.}\\
In our TwinsCoin implementation, we are testing that the mempool which is empty before the test should be able to store a number of transactions up to a maximum specified in settings.

\item \textit{Filtering of valid and invalid transactions from a mempool should be fast}.\\
We are checking that an implementation of mempool is able to filter out an invalid transaction reasonably fast. As processing time is platform-dependent, the test during its instantiation is measuring time to calculate 500,000 blocks of SHA-256 hash. Time to filter out the transaction should be no more than that. 

\item \textit{A transaction successfully added to memory pool should be available by a transaction identifier.}\\

The purpose of this test is to ensure that once a transaction is added to the memory pool, it indeed is available by a transaction identifier. The test simply adds the transaction to the memory pool and then query the transaction by its identifier. The initial transaction is the only correct result of the compound operation. 

\end{itemize}

\item \textit{History Tests.}\\

History is an abstract data structure which records persistent modifiers and is same as what a blockchain is in Bitcoin. Since history is an integral part of a node view, it is important to check if an implementation of history acts correctly. A consensus protocol aims at establishing a common history for all the nodes on the network.

A persistent modifier is the main building block of a history and is used to update the history and a minimal state. As soon as a valid modifier gets appended to history, a node's local view changes in the sense that the history is updated, possibly along with the minimal state.

\begin{itemize}[\IEEEsetlabelwidth{Z}]
\item \textit{A syntactically valid persistent modifier should be successfully applied to history and available by its identifier after that}.\\

A syntactically valid persistent modifier, once applied to the history, should be available by its unique identifier. The importance of this test comes from the fact that it is of utmost importance for the client implementation to tell the difference between the modifiers that have been appended to the history from those that have not been added. For this purpose, the unique identifier of the modifier can be used to query the history to know whether the modifier has been added to the history of not.

\item \textit{A syntactically invalid modifier should not be able to be added to history.}\\
A syntactically invalid modifier should not be added to history and hence, should not be available by id.

\item \textit{Application of a syntactically invalid modifier (inconsistent with the previous ones) should be unsuccessful}.\\
When the unique identifier of an invalid modifier is queried from history, it should always return empty result which shows that the invalid modifier has not been added to the history.

\item \textit{Once a syntactically invalid modifier is appended to history, it should not be available in history by its identifier.}\\
A syntactically invalid modifier is one which is inconsistent with the present view of history. Only syntactically valid modifier is eligible to be applied to history. To filter out invalid modifiers before adding them to history, we propose this test. We generate a random invalid modifier and attempt to add it to history. Since it is invalid, history should not add it and hence, it should not be available by identifier when queried from history. Some examples of generating invalid modifiers are - generate false nonce which doesn't satisfy the puzzle, generate false signatures which validate to false when checked with the public key of the signer, etc.

\item \textit{Once a syntactically valid modifier is appended to history, the history should contain it.}\\
This test ensures that all the valid modifiers are correctly appended to the history meaning that they are indeed capable of changing the node's view. Since the history appends valid modifiers only, only should a valid modifier by available by identifier when queried from history.

\item \textit{History should be able to report if a modifier is semantically valid after successfully appending it to history.}\\
This purpose of this test if to check the consistency in the working of minimal state and the history. Whenever a node's view is updated, both the history and the state should be updated in harmony and consistently with one another. Since a semantically valid modifier in one which is valid based on a node's view of minimal state, history should be able to correctly report the validity of semantically valid modifier after appending it. This also implies that the history should always report the semantic validity of an appended modifier as "valid" since only totally valid modifiers should be able to append to history. On a side-note, this test also checks that history indeed is checking that only totally valid modifiers, and not just syntactically valid ones (the modifiers valid based on history) are appended to history.
\end{itemize}

\item \textit{Minimal State Tests.}\\

In case of an unwanted event, a rollback has to be performed which essentially rolls the system back to a previous "safe" (changes the current system state to a state in the past which has been verified and is seemingly correct) state and hence recovers the system from the unwanted state. A very common example where a rollback might be necessary is when a system has to recover from a fork issue.

\begin{itemize}[\IEEEsetlabelwidth{Z}]

\item \textit{Application and rollback should lead to the same minimal state.}\\

In this test, a semantically persistent modifier $m$ is generated and applied to a current state $S$. Due to this application of the modifier, a new minimal state $S'$ is to be obtained from $S$. After the modifier is applied successfully to the history, a rollback is performed to take the system back to state $S$ from the current state $S'$. The test now checks that the state to which the system comes after the rollback is indeed the state $S$ by checking an identifier of the new state after rollback is the same as the identifier of the original state.\\

We now use this test to explain how we generate a semantically valid modifier for the Twinscoin client. Before proceeding, we define the structure of a transaction $t$. A simple transaction is usually represented as the map $T : UTXO \to UTXO$, where $UTXO$ is the set of all the unspent transaction outputs or \textit{boxes}. A box can be considered as a tuple ($pubkey, amount$) where $pubkey$ is the public key of the account of the node to which this box belongs to and $amount$ refers to the monetary (in case of cryptocurrencies) amount which this box holds in the name of the $pubkey$ in the first half of the tuple. A transaction uses some ($\geq 0$, 0 in the case of the rewarding transaction which is present at the end of each block and which rewards the miner) unspent boxes and generates new boxes with a constraint that sum of the amount of all the boxes used in the transaction is equal to the sum of amount of the boxes output by the transaction except the rewarding transaction for which this constraint doesn't apply. Once the new boxes have been added to the UTXO set, the old boxes which were input to the transaction are removed from the UTXO set to prevent double spending. This addition and removal of boxes from $UTXO$ set has to be done atomically in order to avoid inconsistencies in the system. Suppose a node $A$ wants to send $x$ amount to a node $B$, then the transaction for this purpose will use some boxes with $A$s public key on them which sum up to an amount $y \geq x$ and output the boxes $b_1$ and $b_2$ such that $b_1 = (B, x)$ which has $B$s public key and an amount $x$ whereas $b_2 = (A, y-x)$ will have $A$s public key and an amount of $y-x$, if $y>x$. Now $B$ has received a new box $b_1$ which belongs to him and this box now sits inside the set $UTXO$ until the point when $B$ uses this box an one of the inputs to a future transaction. Readers should note that for more readability we will represent a transaction $T$ by a tuple ([$in_1, in_2, ...$], [$o_1, o_2, ...$]) where $in_i$ denotes input boxes to the transaction and $o_i$ denotes the output boxes of the transaction.\\
Along with checking that the $id$s are same, it also checks that the components of the new state are also rolled back and not just the $id$ number got rolled back. For this, we generate the modifier $m$ in the following way:
\begin{itemize}
\item Generate a pair of transactions ($t_1, t_2$) where $t_1 = ([b_{1}], [b_{2}])$ and $t_2 = ([b_{2}], [b_{3}])$. This notation means that the first transaction $t_1$ uses a box $b_1$ as its input and then outputs a box $b_2$ which is then used by the second transaction as its input. In the above setting, we select the first input box $b_1$ randomly from the set $UTXO$ and finally output the box $b_3$ from $t_2$ which also just generates a random box ($b_3$). The generated transaction pair has to be valid in the sense that it should only use valid unspent boxes from $UTXO$ and satisfy the constraint that the sum of amounts of all the input boxes should be equal to the sum of amounts of the output boxes. The main caveat here is that the second transaction of the pair should use the output box of the transaction of the first transaction of the pair.
\item Now we generate a pair of modifiers ($m_1, m_2$) and include both of these transactions from the pair above in the respective modifiers.
\end{itemize}

Once the custom modifiers are generated, $m_1$ (first half of the pair) is appended to the history and the system moves from state $s_1$ to the state $s_2$. As mentioned before, the transaction $t_1$ from the pair uses a random box $b_1$ from the $UTXO$ of state $s_1$ and when the system moves to the state $s_2$, the $UTXO$ gets added with the box $b_2$ and $b_1$ is removed from the set. Once the state change happens, we append $m_2$ (second half of the modifier pair) to history progressing the system to state $s_3$. Since $m_2$ contains the transaction $t_2$ which takes as input $b_2$, when the system moves to $s_3$, $b_2$ is removed from $UTXO$ and $b_3$ is added.\\

It will now become clear why we generated pairs of transactions and modifiers in the way defined above. Finally, we perform a rollback from state $s_3$ to $s_2$ which should mean that once the rollback is successful, the box $b_2$ should come back to the set $UTXO$ and should be available by $id$ whereas the box $b_3$ should now not be present inside the $UTXO$ set anymore. Both of these checks tell us that the rollback was performed correctly and the system indeed came back to the previous state with all its components. The reason that we generated the pairs of transactions above is because it helps us in easily checking by $id$ if $b_2$ has returned to the $UTXO$ set since we generated $b_2$ ourselves and know its $id$ already. This makes testing easy and transparent. We would like to mention that this test could also be performed without creating the pairs that we have mentioned above.

\item \textit{Application of a valid modifier after a rollback should be successful.}\\
As the previous test aimed at checking that the components of a state are recovered after a rollback happens, it would be quite wrong to think that it should be the only test that is necessary to check if the rollback system performs as expected. It is also equally important that after rollback the system performs normally, as it would perform if the rollback would have never happened. To check this property to certain degree, we propose this test. In this test, we check that after a rollback has happened the system becomes stable again and any new valid modifier which is now added to the history is actually recorded and hence should be available if queried from history. This test ensures that after recovering from a rollback the system performs normally and can resume its functioning without any issues. It hence ensures that a continuity is maintained after a rollback.
\end{itemize}

\item \textit{Common Component Tests.}\\

\item \textit{Application of the same modifier twice should be unsuccessful.}\\

This test checks that a persistent modifier should not be added more than once. Since if a modifier is added twice, all the transaction inside the modifier will be double spent, so an implementation of a blockchain system should prevent addition of a modifier twice. In this test, we generate a modifier, then append it to history once and on the second application of the modifier again to history, the history should return an unsuccessful addition.


\item \textit{Node View Holder Tests.}\\

As was mentioned in Section~\ref{sec:scorex}, node view holder is the central component of a blockchain node which is responsible for atomically updating the quadruple \textit{<history, minimal state, vault, memory pool>}. The update could be triggered by whether a persistent modifier of a transaction coming in. 


\begin{itemize}[\IEEEsetlabelwidth{Z}]

\item \textit{Totally valid modifier should successfully update the minimal state and the history.}\\
We recall that a totally valid modifier is a persistent modifier which is valid for both the history and the minimal state, so it is applicable to both of them. In this test we are sending a random totally valid persistent modifier to the node view holder component and then check that history contains it, and the version of the minimal state is equals to the modifier's identifier.

\item \textit{Modifier application should lead to new minimal state whose elements' set intersection with previous ones is not complete.}\\

Whenever a valid modifier is applied, this should lead to a new minimal state taking into account the necessary changes introduced by the transaction inside the modifier. Readers should note that the cardinality of these changes when represented as a set should not be 0, meaning that there should be some changes to the minimal state since the history and the node's view changed on application of the modifier. This implies that the new minimal state should not be completely same as the previous minimal state before the application of the modifier. If they both were same, this would mean that the application of the modifier did not change anything and hence the modifier was empty (not possible). Therefore, with this test we check that the intersection of minimal state before the application of a modifier and after its application is not complete i.e. not all elements in both the minimal states are same.
\end{itemize}
\end{enumerate}



%- Valid box should be successfully applied to state, it's available by identifier after that.
%- State should be able to generate changes from valid block and apply them.
%- Wallet should contain secrets for all it's public propositions.
%-
%- Transactions once added to a block should be removed from the local copy of mempool.
%- Minimal state should be able to add and remove boxes based on received transaction's validity.
%-
%- 
%- %- BlockchainSanity test that combines all this test.

% !TEX root = laws.tex

\section{Conclusion}
\label{sec:conclusion}

In this paper we propose to improve quality of blockchain protocol implementations via exhaustive property-based testing. For generic abstract modular Scorex framework, we have implemented a suite of property-based tests. The suite consists of 59 tests checking different properties of a blockchain system. To run the suite against a concrete blockchain protocol client, developers of the client need to provide generators for random objects used by the protocol. The suite is checking properties against the implementation by using random samples. We used Twinscoin implementation provided with Scorex as an example of a concrete blockchain using our testing kit. In the paper we provide many examples of the tests.  


\bibliography{sources.bib}


\end{document}
