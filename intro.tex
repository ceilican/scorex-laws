% !TEX root = laws.tex

\section{Introduction}

A blockchain-based cryptocurrency system consists of a set of policies and protocols, such as the consensus protocol, the monetary policy for token emission, the rules for transaction processing, a peer management protocol and network communication protocols. A \emph{node}, or a {\em client}, is a software implementation of the protocols. Usually not all the details of all the protocols are rigorously specified. Instead, typically there is a reference client implementation that acts as the  definition of the protocols for other implementations. A notable exception here is Ethereum, where the Yellow Paper~\cite{ethyp} tries to define all the details of a client implementation. In Bitcoin, however, the reference implementation Bitcoin Core is considered to be standard, and any alternative implementation is expected to reproduce its behavior, and even its bugs~\cite{bitbugs}.

Repeated testing of even the most carefully written and designed system is crucial to expose hidden vulnerabilities in the developed system which might miss the eye of the developers. Such tests should be performed regularly in order help ensure reliability, security and performance of the system. Furthermore, in the case of Bitcoin and other cryptocurrencies, an error in a client implementation could be utterly costly and hard to fix. For example, the famous value oveflow bug in Bitcoin~\cite{overflow} caused a fork of more than 50 blocks~(more than 8 hours) and required a soft fork~(for the majority of miners to upgrade) to be fixed. With an increasing demand for the development of more clients for existing as well as new alternative blockchain systems and cryptocurrencies, such costly bugs can be expected to become more common and problematic and testing can be expected to become one of the most important parts of the software development lifecycle for blockchain systems.

This paper addresses these trends by:
\begin{enumerate}
\item proposing a generic property-based testing framework that can be easily plugged into the implementation of a concrete client;
\item describing some of the essential properties which ought to hold true for any cryptocurrency implementation.
\end{enumerate}

