% !TEX root = laws.tex

\section{Introduction}

A blockchain-based cryptocurrency is defined by a set of protocols, such as the consensus protocol, agreement on tokens emission rules, deterministic transactions processing protocol, and so on. A node, or a {\em client} is software implementation of the protocols. Even more, usually not all the details of all the protocols are specified, so there is some reference client implementation which code acts as a protocols definition for other implementations. The notable exception here is Ethereum, which Yellow Paper~\cite{ethyp} is trying to define all the details of a client implementation. In Bitcoin, the reference implementation Bitcoin Core is considered to be standard, so an alternative implementation must reproduce its behavior, and even its bugs~\cite{bitbugs}. 

An error in a client implementation would be utterly costly and hard to fix. For example,the famous value oveflow bug~\cite{overflow} caused a fork of more than 50 blocks~(more than 8 hours) and did require soft fork~(so majority of miners to upgrade). On other hand, there is an increasing demand for the development of more blockchain protocols and clients. 


\subsection{Our Contribution}

In this paper we report on the design and implementation of a suite of abstract property tests which are implemented in the Scorex framework to ease checking whether a blockchain client satisfies specified laws. A developer of a concrete blockchain system just needs to implement generators of random test inputs~(such as blocks and transactions), and then the testing system will extensively check properties against multiple input objects. We have implemented 59 property tests. We have integrated the tests into a prototype implementation of the TwinsCoin~\knote{cite} cryptocurrency, which has two types of blocks. \knote{enhance} 
