% !TEX root = laws.tex

\section{Introduction}

A blockchain-based cryptocurrency is defined by a set of protocols, such as the consensus protocol, agreement on tokens emission rules, deterministic transaction processing protocol, and so on. A node, or a {\em client} is software implementation of the protocols. Usually not all the details of all the protocols are rigorously specified, so there is some reference client implementation which code acts as a protocols definition for other implementations. The notable exception here is Ethereum, which Yellow Paper~\cite{ethyp} is trying to define all the details of a client implementation. In Bitcoin, the reference implementation Bitcoin Core is considered to be standard, so an alternative implementation must reproduce its behavior, and even its bugs~\cite{bitbugs}. 

An error in a client implementation would be utterly costly and hard to fix. For example, the famous value overflow bug~\cite{overflow} caused a fork of more than 50 blocks~(more than 8 hours) and did require soft fork~(so majority of miners to upgrade). On other hand, there is an increasing demand for the development of more blockchain protocols and clients. The solution we are reporting progress on in this paper is to develop a generic testing framework which could be easily plugged into a concrete system implementation.  
